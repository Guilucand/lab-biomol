\documentclass{article}

\usepackage{graphicx} % Required for the inclusion of images

\setlength\parindent{0pt} % Removes all indentation from paragraphs
\usepackage[utf8x]{inputenc}

\usepackage{times} % Uncomment to use the Times New Roman font
\usepackage{siunitx}



\title{Utilizzo delle colture cellulari umane in analisi molecolari e morfologiche \\ (Esercitazione n° 6)} % Title

\author{Litterini S. \\Giulio B. \\Cracco A.\\Buzzolan T. } % Author name

\date{20 Aprile 2018} % Date for the report

\begin{document}

\maketitle


\section{Sommario}

\subsection{Scopo}

L'obiettivo di questa esperienza e' ...


\subsection{Cenni teorici}

\section{Materiali utilizzati}

\begin{itemize}
\item Guanti in lattice
\item Pasteur in plastica monouso
\item Cell scraper
\item Falcon da utilizzare come scarto
\item Provette Eppendorf (2mL)
\item Micropipette (100-1000  e 2-200 microlitri  )
\end{itemize}


\section{Soluzioni utilizate}
\item Cristalvioletto in Eppendorf da 1,5ml
\item RNAlater in Eppendorf da 1,5ml
\item 5ml di PBS 10X + 45ml ddH$_2$O, in provetta Falcon 50ml

\begin{itemize}

\item miniprep del pUC18
\item cellule competenti
\item LB liquido

\end{itemize}



\section{Procedimento}

\begin{enumerate}

\item Aspirare


\end{enumerate}


\section{Risultati e Conclusioni}

Tramite questa Procedura si è potuto inserire all'interno delle nostre cellule batteriche di E.coli rese competenti, i plasmidi pUC18 contenenti i geni di mio interesse da clonare, o da esprimere.


\end{document}
