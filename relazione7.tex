\section{\LARGE{Allestimento di colture cellulari}}

\vspace{0.6cm}

\subsection{Sommario}

\subsubsection{Scopo}

Quest'esperienza in laboratorio ha come scopo l'allestimento di una coltura cellulare
su un Multiwell plate per poter effettuare esperimenti.

\subsubsection{Cenni teorici}

La creazione di colture cellulari \`e tra le attivit\`a pi\`u comuni e importanti
effettuate in un laboratorio di biologia molecolare. Le colture permettono di
far replicare le cellule, facilitando l'estrazione di RNA, DNA e proteine oltre a
permettere la selezione delle cellule trasformate grazie all'aggiunta di antibiotico
al terreno.

\subsubsection{Materiali utilizzati}

\begin{itemize}
	\item Guanti in lattice
	\item Falcon da 50ml
	\item Cell-scraper
	\item Flask con cellule
	\item Centrifuga
	\item Multiwell plate
\end{itemize}

\subsubsection{Soluzioni utilizzate}

\begin{itemize}
	\item PBS
	\item Terreno DMEM
	\item FBS
	\item Cocktail di antibiotici-antimicotici
\end{itemize}

\subsection{Procedimento}

\subsubsection{Preparazione del terreno di coltura}

\begin{itemize}
	\item Aliquotare 10ml di terreno DMEM sotto cappa sterile
	(per evitare contaminazioni) in Falcon da 50ml.
	\item Supplementare il terreno aggiungendo il 10\% di FBS.
	Il siero fetale bovino e' necessario per fornire alle cellule sostanze non
	presenti nel terreno di coltura, come proteine plasmatiche,
	fattori di crescita, chelanti, vitamine ed elettroliti.
	\item Aggiungere cocktail di antibiotici-antimicotici in concentrazione 1\%
	c\`o permette la selezione dei batteri trasformati dai plasmidi
\end{itemize}

\subsubsection{Staccare le cellule mediante cell-scraper}
\begin{itemize}
	\item Eliminare il terreno esausto.
	\item Lavare con 5ml di PBS, agitando delicatamente.
	\item Ripetere il lavaggio con PBS.
	\item Bagnare tutta la superficie della flask con 10ml di PBS e staccare
	delicatamente le cellule con lo scraper. Se viene esercitata troppa forza le
	cellule rischiano di lisarsi.
	\item Prelevare le cellule in PBS e spostarle in una Falcon da 15ml.
	\item Centrifugare per 15' a 500g.
\end{itemize}

\subsubsection{Spostare le cellule nel Multiwell plate}
\begin{itemize}
	\item Una volta ottenuto il pellet eliminare il surnatante sotto cappa sterile.
	\item Risospendere il pellet in 1,5ml di terreno completo.
	\item Aliquotare 500$\mu$l di cellule per pozzetto.
	\item Riempire i pozzetti.
\end{itemize}

\subsection{Risultati e Conclusioni}

Durante questa esperienza abbiamo imparato a creare un terreno di coltura
e spostare le cellule in esso, tramite l'utilizzo del cell-scraper.
Abbiamo inoltre visto come utilizzare il Multiwell plate come supporto per la
nostra coltura.
